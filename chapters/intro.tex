

\فصل{مقدمه}

در ابتدا در مورد مسأله‌ی آنالیز بقا صحبت خواهیم کرد و آن را شرح خواهیم داد. سپس در مورد بیماری سرطان دهان که در این پایان‌نامه به آن پرداخته‌ایم، صحبت خواهیم کرد و از اهمیت و کاربرد آنالیز بقا در مورد این بیماری خواهیم گفت. سپس کلیتی از آنچه در این پایان‌نامه بررسی شده و اهداف این تحقیق بیان می‌شود. نهایتاً در انتهای این فصل، ساختار پایان‌نامه و کلیتی از محتوای فصل‌های آن ارائه خواهد شد.


\قسمت{تعریف مسئله‌ی آنالیز بقا}

مسأله‌ی آنالیز بقا در واقع تعیین مدت زمان میان تشخیص یک بیماری و مرگ ناشی از آن است \مرجع{r4}. هدف این است که بتوانیم براساس داده‌هایی که از یک بیمار داریم تخمینی از مدت زمانی که او زنده می‌ماند و یا شدت بیماری داشته باشیم. این مسأله مدت زیادی است که بررسی می‌شود و روش‌ها و شیوه‌های گوناگونی برای آن وجود دارد. در گذشته روش‌های کلاسیک استفاده می‌شد که مبنای آماری داشتد، اما اخیراً روش‌های متنوع یادگیری ماشین هم برای این هدف به کار گرفته می‌شوند. مسأله آنالیز بقا اکثراً بدین شکل بررسی می‌شود که داده‌های تعدادی از بیماران موجود است و مدل ما، چه به شکل آماری چه با کمک یادگیری ماشین، اطلاعاتی از داده‌ها استخراج کرده و پیش‌بینی را انجام می‌دهد. سپس مدل می‌تواند با گرفتن داده‌های بیماران جدید، تخمینی از مدت زمان بقای آن‌ها ارائه دهد.

نکته‌ای که در خصوص تحلیل مسأله بقا وجود دارد این است که داده‌هایمان به شکل سانسور شده\زیرنویس{\lr{Censored Data}} هستند. به این معنی که ما در داده‌هایمان از بیماران:
\شروع{فقرات}
\فقره
تعدادی بیمار داریم که فوت شده‌اند و مدت زمان بقایشان پس از تشخیص بیماری و انجام آزمایشات پزشکی را می‌دانیم.
\فقره
تعدادی بیمار داریم که فوت نشده‌اند یا اطلاعات فوت آنها در دسترس نیست و ما صرفاً تاریخ آخرین مراجعه‌ی آنها را داریم و می‌دانیم که تا زمان آخرین مراجعه زنده بوده‌اند.
\پایان{فقرات}

این موضوع سبب می‌شود که مسأله آنالیز بقا کمی متفاوت‌تر از سایر مسائل موجود شود و راهکارها و مدل‌های متفاوتی برای حل این مسأله طراحی و به کار گرفته شوند. 

\قسمت{سرطان دهان و اهمیت تحلیل بقا}

سرطان دهان از سرطان‌های به نسبت رایج می‌باشد. در سال $2020$ میلادی، تقریباً $2$ درصد سرطان‌هایی که در دنیا تشخیص داده شدند، سرطان دهان بودند. بیش از $370$ هزار نفر در آن سال مبتلا به سرطان دهان تشخیص داده شدند و $170$ هزار نفر نیز براساس ابتلا به سرطان دهان فوت کردند \مرجع{r1}. این سرطان خطرناک و کشنده است و تنها کمی بیشتر از نصف افرادی که به آن مبتلا می‌شوند می‌توانند بیش از $5$~سال زنده بمانند \مرجع{r2}. براساس این آمار، می‌توان فهمید که نرخ مرگ سرطان دهان بالاست و  فرصت طولانی برای درمان و معالجه آن وجود ندارد و ممکن است سرطان به سرعت رو به وخامت برود و سبب فوت بیمار گردد. لذا بهره‌گیری از شیوه‌های درست درمانی برای مقابله با سرطان دهان بسیار مهم است و درمان‌های اشتباه و نامناسب می‌تواند فرصت کوتاه درمان را تلف کند. بیماری که بقای کوتاه‌تری برایش پیش‌بینی شده است، نیاز به اجرای روش‌های درمانی پرریسک‌تر و سنگین‌تر دارد؛ در سمت مقابل برای بیماری که بقای طولانی پیش‌بینی شده است احتمالاً درمان‌های سبک‌تر هم می‌تواند مفید واقع شود \مرجع{r3}. پس پیش‌بینی بقای بیماران سرطان دهان و نیز تحلیل اهمیت ویژگی‌هایی که از بیماران استخراج شده است و شناخت ویژگی‌هایی که می‌توانند در پیش‌‌آگاهی به ما کمک بیشتری کنند حائز اهمیت است.


\قسمت{داده‌ها}

داده‌هایی که در این پایان‌نامه مورد استفاده قرار می‌گیرند، داده‌هایی هستند که از بیماران سرطان دهان بیمارستان شهید بهشتی به دست آمده‌اند.  این داده‌ها در برگیرنده‌ی اطلاعات~$526$~بیمار مبتلا به سرطان دهان هستند. برای هر بیمار $22$ ویژگی به دست آمده که ما از این ویژگی‌ها برای تحلیل بقای بیماران استفاده می‌کنیم. در فصل‌‌های آتی به بیان جزئیات ویژگی‌ها خواهیم پرداخت.

\قسمت{اهداف تحقیق}

در این پایان‌نامه، قصد داریم گونه‌های مختلف روش‌های تحلیل بقا را روی داده‌های بیماران سرطان دهان که در اختیار داریم، بررسی کنیم و عملکرد آن‌ها را با یکدیگر مقایسه کنیم. تعداد زیادی روش بررسی خواهد شد و هر یک از آنها از نظر معیارهای گوناگونی که برای سنجش کارایی روش‌های تحلیل بقا وجود دارد ارزیابی می‌شوند. همچنین به بررسی ویژگی‌های مختلف در داده‌هایمان می‌پردازیم و سعی می‌کنیم ویژگی‌ها و عواملی که در تشخیص بقای بیماران موثرتر هستند را شناسایی کنیم. به بیان بهتر می‌خواهیم اهمیت ویژگی‌ها را متوجه شویم. مشخص شدن ویژگی‌های مهم‌تر می‌تواند در معاینات و مطالعات پزشکی هم موثر واقع شود.


\قسمت{ساختار پایان‌نامه}

این پایان‌نامه شامل پنج فصل است. 
فصل دوم دربرگیرنده‌ی تعاریف و مفاهیم اولیه‌ی مرتبط با پایان‌نامه است. همچنین در خصوص داده‌هایمان و آماده‌سازی آنها برای آزمایشات توضیحاتی ارائه می‌شود.
در فصل سوم، به معرفی و بیان روش‌های گوناگون حل مسأله تحلیل بقا خواهیم پرداخت. همچنین چند مورد از معیارهایی که در مسأله تحلیل بقا برای سنجش کارایی و دقت مدل‌ها موجود است را معرفی می‌کنیم.
در فصل چهارم، ایده‌هایی که برای تعیین اهمیت ویژگی‌ها در مدل‌های مختلف آزمایش کردیم را شرح می‌دهیم. با کمک این ایده‌ها سعی کردیم که ویژگی‌های مهم‌تر در مسأله تحلیل بقا را شناسایی کنیم.
در فصل پنجم نتایج به کارگیری مدل‌های معرفی‌شده روی داده‌هایمان را بیان می‌کنیم و به مقایسه مدل‌های مختلف می‌پردازیم. همچنین نتایج تحلیل اهمیت ویژگی‌ها را شرح می‌دهیم و ویژگی‌هایی که از نظر مدل‌های مختلف حائز اهمیت بیشتری هستند را مشخص می‌کنیم. 
فصل ششم به نتیجه‌گیری و پیش‌نهادهایی برای کارهای آتی خواهد پرداخت.

