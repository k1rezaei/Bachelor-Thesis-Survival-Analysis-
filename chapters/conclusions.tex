
\فصل{نتیجه‌گیری}

در این پایان‌نامه ما به روش‌های حل مسأله‌ی آنالیز بقا پرداختیم و دسته‌بندی کلی‌ای از شیوه‌های کلاسیک و مبتنی بر یادگیری ماشین ارائه دادیم. سپس تعدادی از مهم‌ترین شیوه‌ها را روی مجموعه داده‌هایمان بررسی کردیم و به مقایسه آنها با یکدیگر براساس معیارهای موجود پرداختیم. بنابر نتایج آزمایشات، مدل‌های \lr{Logistic Hazard} و جنگل بقای تصادفی بیشترین دقت و بازدهی را داشتند و بهتر از بقیه‌ی مدل‌های موجود عمل کردند.

همچنین در ادامه‌ی کار سعی کردیم که ویژگی‌های مهم را شناسایی کنیم. برای این کار از ویژگی‌های جنگل بقا و ماشین تقویت گرادیان استفاده کردیم و نیز شیوه‌ای برای بررسی اهمیت ویژگی‌ها برای مدل \lr{Logistic Hazard} ارائه دادیم. بر مبنای آزمایش‌هایمان، متوجه شدیم که ویژگی‌های همچون سن، زمان عود، انجام جراحی، انجام شیمی‌درمانی، سایز تومور و پرتودرمانی از مجموعه ویژگی‌هایی هستند که در آنالیز بقا در مورد بیماران سرطان دهان حائز اهمیت‌اند و نبود آن‌ها در مجموعه‌ی ویژگی‌ها می‌تواند باعث شود که مدل‌های بالا به دقت پایینی برسند.

مدل‌های فراوان دیگری نیز در زمینه‌ی آنالیز بقا وجود دارند که می‌توانستند در این پروژه بررسی شوند. همچنین امروزه مدل‌های بسیار متفاوتی مبتنی بر شبکه‌های عصبی ارائه شده‌اند که آن‌ها نیز می‌توانستند بررسی شوند و ما تنها به $3$ مورد از آن‌ها پرداختیم. همچنین در خصوص تعیین اهمیت ویژگی‌ها، می‌توان از ایده‌هایی که مستقیماً به ساختار شبکه‌های عصبی مربوط به این مدل‌ها وابسته هستند (مثل گرفتن مشتق نیست به ویژگی‌ها و ...) نیز استفاده کرد و بیشتر در مورد اهمیت ویژگی‌ها اطلاعات به دست آورد.

