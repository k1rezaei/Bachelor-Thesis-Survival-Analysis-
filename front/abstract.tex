
% -------------------------------------------------------
%  Abstract
% -------------------------------------------------------


\pagestyle{empty}

\شروع{وسط‌چین}
\مهم{چکیده}
\پایان{وسط‌چین}
\بدون‌تورفتگی
پیش‌بینی شدت یک بیماری براساس وضعیت فعلی بیمار در به کارگیری شیوه‌ها و برنامه‌های درمانی مناسب بسیار حائز اهمیت است. توسعه مدل‌هایی که بر مبنای داده‌های کلینیک‌های درمانی بتوانند شدت بیماری در یک بیمار را معین کنند، می‌تواند در این راستا مفید واقع شود.   این داده‌ها اکثراً در برگیرنده تعداد زیادی از ویژگی‌های بیماران هستند که غالباً سانسورشده و بخشی از اطلاعات ناقص می‌باشد. این عوامل سبب می‌شوند که برای آنالیز بقا و استخراج اطلاعات از ویژگی‌های بیماران نیاز به شیوه‌هایی داشته باشیم که توانایی حل این چالش‌ها را داشته باشند. اخیراً شیوه‌های مبتنی بر یادگیری ماشین به خاطر توانایی‌شان در یادگیری توابع پیچیده از داده‌های در ابعاد بالا توجهات زیادی در زمینه‌های مختلف از جمله آنالیز بقا به خود جلب کرده‌اند. در این پایان‌نامه ما به بررسی گونه‌های مختلف الگوریتم‌های یادگیری ماشین در زمینه آنالیز بقا می‌پردازیم و آنها را روی داده‌هایمان آزمایش کرده و نتایج مدل‌های مختلف را با یکدیگر مقایسه می‌کنیم.  مدل \lr{Logistic Hazard} و جنگل بقای تصادفی بهترین عملکرد را در بین مدل‌های آزمایش شده دارند. همچنین با کمک این مدل‌ها، ویژگی‌هایی که اهمیت بیشتری در پیش‌بینی شدت بیماری و بقای بیمار دارند را به دست می‌آوریم که به نوعی در تفسیر مدل‌های مربوطه به ما کمک می‌کنند.

\پرش‌بلند
\بدون‌تورفتگی \مهم{کلیدواژه‌ها}: 
یادگیری ماشین، آنالیز بقا، شبکه‌عصبی، انتخاب ویژگی، تفسیر‌پذیری	
\صفحه‌جدید
